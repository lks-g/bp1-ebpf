\section{Teoretické základy eBPF}
Táto kapitola je vypracovaná podľa internetových stránok ebpf.io, kernel.org, oracle.com, sysdig.com a knihy od spisovateľky Liz Rice~\cite{eBPF, kernel, oracle, history, book1, B.Gregg}

V súčasnej dobe sa operačné systémy stávajú čoraz komplexnejšími, s rastúcim počtom aplikácií a služieb, ktoré sa na týchto systémoch spúšťajú, 
rastie aj potreba zabezpečenia pred rôznymi bezpečnostnými rizikami a hrozbami. Jedným z nástrojov, ktorý sa používa na zlepšenie bezpečnosti v 
operačných systémoch Linux, je eBPF (\emph{extended Berkeley Packet Filter}). Je to flexibilný a mocný nástroj, ktorý sa čoraz častejšie používa na 
rôzne účely v operačných systémoch, ako je napríklad filtrovanie sieťovej aktivity, optimalizáciu výkonu operačného systému, 
alebo detegovanie útokov a škodlivého kódu. Taktiež je čoraz častejšie používaný pre implementáciu rôznych nástrojov slúžiacich na analýzu dát 
a monitorovanie systémov, ako napríklad nástroj \emph{perf}, ktorý umožňuje profilovanie výkonu aplikácií v operačnom systéme. Pre maximálny výkon a 
bezpečnosť systému je dôležité, aby bol eBPF správne nakonfigurovaný a používaný pretože môže spôsobiť veľké bezpečnostné rizíká pri zlom zaobchádzaní.

\subsection{História eBPF}
História eBPF sa začína s technológiou nazývanou \emph{Berkeley Packet Filter} v skratke BPF, ktorá vznikla 19. Decembra 1992 kedy ju opísali Steven McCanne a Vano Jacobson.~\cite{bpf} 
BPF bol navrhnutý ako jednoduchý jazyk pre filtrovanie sieťových paketov, ktorý bol implementovaný ako rozšírenie jadra operačného systému. Jeho hlavným cieľom 
bolo umožniť používateľom filtrovať sieťové pakety bez nutnosti používať externé nástroje ako napríklad \texttt{tcpdump}. 
BPF bol úspešne implementovaný v operačných systémoch ako \emph{BSD} a Linux, a stal sa jedným z kľúčových nástrojov pre sieťovú diagnostiku a analýzu. 
Avšak, s rastúcimi požiadavkami na funkcionalitu a výkon, bolo potrebné rozšíriť BPF o ďalšie možnosti. Začiatkom 21. storočia sa začal vývoj zameraný na vylepšenie technológie BPF. 
Nová verzia nazvaná eBPF (\emph{extended Berkeley Packet Filter}) vznikla ako rozšírenie k BPF, ktoré bolo navrhnuté tak, aby poskytlo viac funkcií a umožňovalo vykonávať 
komplexnejšie filtračné skripty na sieťových paketoch. Tento nástroj bol prebratý Linuxovou komunitou, kde sa stal veľmi populárnym pre rôzne účely a v roku 2014 bol 
implementovaný do jadra Linuxu. Medzi hlavné rozdiely BPF a eBPF patrí podpora pre \emph{x86} a \emph{arm} architektúry, možnosť spustiť aplikácie v jadre operačného systému 
a možnosť vykonávať viac operácií ako len filtrovanie sieťových paketov. V súčasnosti eBPF umožňuje používateľom načítať a spustiť vlastné programy vo 
virtuálnom priestore v rámci jadra operačného systému. To znamená, že môže rozšíriť alebo dokonca upraviť spôsob, akým sa jadro správa 
bez zmeny zdrojového kódu jadra.~\cite{book1} Programy spustené týmto spôsobom sú schopné monitorovať systém, zhromažďovať metriky, a dokonca aj vykonávať 
rôzne úlohy ako napríklad sledovanie a upravovanie sieťovej aktivity. Vďaka týmto rozšíreniam sa eBPF stal veľmi flexibilným nástrojom pre 
riešenie rôznych problémov v oblasti sieťovej diagnostiky, monitorovania systému a izolácie kontajnerov.

\subsection{Princípy eBPF}
Z úvodu vieme, že eBPF je veľmi silný nástroj pre vývojárov operačných systémov, ktorý poskytuje flexibilitu a vysoký výkon. 
Umožňuje spustenie malých programov priamo v jadre operačného systému. Tieto programy sa nazývajú \emph{"eBPF programy"} a sú napísané v bajtkóde. 
Následujúce podsekcie sú zamerané na hlavné princípy eBPF, funkciu eBPF, vstupné body a spôsoby akými je možné písať samotné eBPF programy. 
Vo všeobecnosti eBPF prináša možnosť jednoduchého a flexibilného prístupu k jadru operačného systému, ktorá umožňuje vytvoriť nástroje pre monitorovanie, 
analýzu a riadenie systému bez potreby úpravy samotného jadra operačného systému. 
Medzi hlavné princípy eBPF patrí: 
\begin{itemize}
    \item \textbf{Bezpečnosť}: všetky eBPF programy sú spúšťané vo virtuálnom prostredí, vďaka čomu nespôsobujú poškodenie systému ani iných programov.
    \item \textbf{Flexibilita}: eBPF programy môžu byť spustené v rôznych častiach jadra, ako napríklad pri sledovaní systémových volaní, 
    pri filtrovaní sieťových paketov, alebo  pri sledovaní využitia pamäte.
    \item \textbf{Vysoký výkon}: eBPF programy sú spustené priamo v jadre, čo v kombinácií s dobre optimalizovaným bajtkódom, 
    špecializáciou programu na konkrétny účel a možnosťou využívať existujúce dátové štruktúry v jadre (\emph{in-kernel datastructures})  
    umožňuje vysoký výkon a malú latenciu.
    \item \textbf{Dynamická úprava kódu}: eBPF programy môžu byť dynamicky upravené, alebo nahradené bez nutnosti reštartovať systém, 
    čo umožňuje rýchlu a jednoduchú úpravu správania operačného systému.
    \item \textbf{Kontrola jadra OS}: eBPF programy umožňujú prístup k interným dátam jadra OS a tým pádom lepšie pochopenie a kontrolu 
    nad chodom celého systému.
\end{itemize}
%\pagebreak

\subsubsection{Ako funguje eBPF} 
eBPF je nástroj, ktorý sa používa na filtrovanie sieťovej aktivity, sledovanie výkonu a implementáciu bezpečnostných politík v operačných systémoch Linux. 
Jeho kľúčovým princípom je používanie \emph{bytecode} interpreteru, pomocou ktorého je schopný vykonávať rôzne operácie. eBPF programy sú riadené udalosťami a 
spúšťajú sa, keď jadro alebo aplikácia prejde určitým bodom ktorý nazývame \hyperref[sec:hooks]{vstupný bod alebo \emph{hook}}.  Po úspešnej kompilácií 
programu do bajtkódu a validácií sa eBPF program nahrá, do jadra Linuxu. Tento program sa potom spustí pri každom prechode zvoleným bodom/hookom. 
eBPF program sa skladá z viacerých častí, ktoré sú zodpovedné za rôzne úlohy. Program môže filtrovať pakety, ukladať informácie do máp, vykonávať výpočty 
a rozhodovať o tom, či má byť paket ponechaný alebo zahodený. Mapou sa označuje jedna z dôležitých funkcií eBPF, ktorá umožňuje ukladať dáta v kerneli 
do hash tabuliek, ktoré sa potom dajú použiť na rôzne účely, napríklad sledovanie výkonu aplikácií, alebo na implementáciu rôznych bezpečnostných politík. 
O eBPF mapách a eBPF pomocných funkciách (\emph{eBPF helpers}) sa budeme viac zaoberať v sekcií \hyperref[sec:architecture]{1.3s}. 
Ďalej tento nástroj umožňuje implementovať rôzne bezpečnostné politiky, ako napríklad firewall, izolácia kontajnerov, a podobne.

\subsubsection{Vstupné body pre eBPF}
\label{sec:hooks}
\emph{eBPF hooks} a vstupné body sú súčasťou eBPF. Slúžia na spustenie eBPF programu pri vyvolaní konkrétnej udalosti v jadre. 
Aj keď sa na prvý pohľad môže zdať, že ide o tú istú vec, nie je to tak. Ich funkcionalita sa môže líšiť v konkrétnom kontexte. 
Vstupné body sú \textbf{miesta v jadre}, kde sa môže eBPF program spustiť. Sú to napríklad:
\begin{itemize}
    \item \textbf{kprobe}: Programy sa spustia, \textbf{pri vstupe} na konkrétnu adresu v pamäti. 
    Táto adresa sa označuje ako \emph{probe point}. Kprobe je veľmi flexibilný, pretože umožňuje 
    sledovanie akcie v jadre aj v užívateľskom priestore.
    \item \textbf{kretprobe}: Spustí eBPF programu \textbf{pri návrate} z konkrétneho kódu v jadre.
    \item \textbf{tracepoint}: Programy sú spustené \textbf{pri dosiahnutí} konkrétnych bodov v jadre, 
    ktoré sú označené ako \emph{tracepoints}. Tracepointy sú najmä využívané na sledovanie systémových 
    udalostí, ako sú volania systémových volaní alebo prijímanie a odosielanie sieťových paketov.
\end{itemize}
\emph{eBPF Hooks}, na druhej strane, sú mechanizmy, ktoré sa používajú na \textbf{pridanie vlastného kódu} do existujúceho kódu v jadre. 
To znamená, že keď sa kód v jadre vykoná, môže sa volať vlastný kód, ktorý bol pridaný cez hook. Hooks sú definované v 
jadre operačného systému a poskytujú eBPF programu prístup k rôznym systémovým udalostiam. Preddefinované hooky zahŕňajú 
systémové volania, vstup a výstup funkcií, sledovacie body jadra, sieťové udalosti a niekoľko ďalších. Pokiaľ pre konkrétnu 
požiadavku neexistuje preddefinovaný hook, je možné vytvoriť \emph{kernel probe} (\texttt{kprobe}) alebo \emph{user probe} (\texttt{uprobe}) 
na pripojenie eBPF programov takmer kdekoľvek v užívateľských aplikáciách alebo jadre.

\subsubsection{Ako sa tvoria programy eBPF}
eBPF programy sa píšu pomocou jazyka s nízkou úrovňou, ako je napríklad C. Keďže eBPF programy sú spustené v jadre systému, je dôležité, 
aby boli bezpečné a nekonfliktné s ostatnými časťami jadra. Preto existujú špeciálne pravidlá a obmedzenia pre písanie eBPF programov, 
ktoré musia byť dodržiavané. V mnohých scenároch sa eBPF nepoužíva priamo, ale nepriamo prostredníctvom projektov ako je napríklad 
\textbf{Cilium} alebo \textbf{KRSI} (\emph{Kernel runtime security instrumentation}). Pre zjednodušenie tvorby eBPF programov vzniklo veľa rôznych 
knižníc a nástrojov, ktoré umožňujú písať eBPF programy v jazykoch s vyššou úrovňou, ako napríklad \textbf{Python}, a potom ich kompilujú do bajtkódu. 
Tieto nástroje pomáhajú vývojárom vytvárať eBPF programy bez potreby hlbokých znalosti jazyka C a pravidiel pre písanie bezpečných eBPF programov. 
Pokiaľ, ale neexistuje abstrakcia vyššej úrovne, je potrebné programy písať priamo. Aj keď je samozrejme možné napísať bajtkód priamo, 
bežnejšou vývojovou praxou je využitie kompilátora, ako je \textbf{LLVM}, na kompiláciu pseudo-C kódu do eBPF bajtkódu. 
Po naprogramovaní a skompilovaní eBPF programu sa následne tento program dá nahrať do jadra linuxu pomocou komunikačného rozhrania \emph{bpf()}.

\subsection{Architektúra eBPF}
\label{sec:architecture}
Táto sekcia je spracovaná podľa ~\cite{eBPF,bpf-doc,book1} a zaoberá sa rôznymi mechanizmami, ktoré eBPF ponúka. 
Pozrieme sa bližšie na preklad kódu cez kompilátor JIT, eBPF mapy a pomocné eBPF funkcie.

\subsubsection{Kompilátor JIT}
Kompilátor JIT (Just-In-Time) je typ kompilátora, ktorý generuje strojový kód počas behu, a nie vopred ako tradičné kompilátory.~\cite{eBPF}
Kompilátor zoberie vloženú reprezentáciu kódu  väčšinou vo forme bajtkódu alebo assembleru a konvertuje ju na strojový kód, ktorý môže byť vykonaný priamo procesorom. 
Hlavnou výhodou používania JIT je, že umožňuje efektívnejšie vykonávanie kódu, pretože tento kompilátor môže optimalizovať kód špeciálne pre prostredie behu, 
napríklad pre konkrétnu architektúru procesora a dostupné pamäťové zdroje. Vďaka tomu sa programy eBPF spúšťajú rovnako efektívne ako natívne skompilovaný 
kód jadra alebo ako kód načítaný, ako modul jadra.~\cite{eBPF} Taktiež má  schopnosť dynamicky prekompilovať kód počas behu programu, vďaka čomu sa dokáže prispôsobiť 
meniacim podmienkam behu a ďalej zvyšovať výkon. V prípade eBPF, JIT kompilácia znamená, že eBPF programy sa kompilujú do strojového kódu priamo v jadre Linuxu, 
keď sa spúšťajú. Tým sa dosiahne vyššia rýchlosť a nižšia spotreba pamäte oproti interpretácii eBPF programu v jadre.

\subsubsection{eBPF Maps}
Dôležitým aspektom eBPF programov je schopnosť zdieľať zhromaždené informácie a ukladať stav.~\cite{eBPF} Na tento účel môžu programy eBPF využívať koncept máp. 
eBPF mapami sa označuje jedna z dôležitých funkcií eBPF, ktorá umožňuje ukladať dáta v kerneli do hash tabuliek, ktoré sa potom dajú použiť na rôzne účely, 
napríklad na sledovanie výkonu aplikácií, alebo na implementáciu rôznych bezpečnostných politík, ako napríklad firewall, izolácia kontajnerov a podobne. 
Tiež sa dajú použiť pre implementáciu rôznych nástrojov pre analýzu dát a monitorovanie systémov. K mapám možno pristupovať z eBPF programov, 
ako aj z aplikácií v používateľskom priestore prostredníctvom systémového volania.

\subsubsection{eBPF Helpers}
eBPF programy nemôžu volať ľubovoľné funkcie jadra. Ak by sa to povolilo, programy eBPF by sa viazali na konkrétne verzie jadra a skomplikovala by sa 
kompatibilita programov. Namiesto toho môžu programy eBPF uskutočňovať volania funkcií do pomocných (\emph{helper}) funkcií. eBPF pomocníci (angl. \emph{Helpers}) 
sú rozšírenia jadra Linuxu, ktoré umožňujú programovať eBPF programy jednoduchšie a bezpečnejšie.~\cite{eBPF} Pomáhajú vytvárať a spravovať eBPF objekty, 
ako sú programy, mapy a tracepointy, a tiež umožňujú komunikovať s jadrom bez nutnosti používať príkazy ako napr. netlink. eBPF Helpre tiež 
poskytujú vysokú úroveň abstrakcie pre rôzne úlohy, ako napríklad sieťovú filtráciu, tj. riadenie prístupu sieťových paketov, a vyhodnocovanie štatistík. 
To umožňuje vývojárom vytvárať aplikácie, ktoré využívajú silné a flexibilné možnosti eBPF, bez nutnosti mať hlboké znalosti jadra Linuxu.
Príklady dostupných pomocných volaní:
\begin{itemize}
\item Prístup k mape eBPF
\item Získať aktuálny čas a dátum
\item Získať kontext procesu/skupiny
\item Manipulácia so sieťovými paketmi a logika presmerovania
\end{itemize}

Súbor obsahujúci všetky dostupné helper funkcie pre eBPF nájdete priamo v manuáli Linuxu. 
\url{https://man7.org/linux/man-pages/man7/bpf-helpers.7.html}

\subsection{eBPF nástroje}
Tvorba eBPF programov je náročný proces. Pre jeho zjednodušenie vzniklo a stále vzniká veľa nástrojov, ktoré napomáhajú k ľahšiemu a hlavne rýchlejšiemu 
používaniu a samotnému vývoju eBPF. Tieto nástroje poskytujú rôzne funkcie, ako napríklad kompiláciu eBPF programov, ladenie eBPF programov, získavanie štatistík o eBPF 
programe alebo vizuálne zobrazovanie dát získaných prostredníctvom  eBPF. V súčasnosti existuje veľa rôznych typov eBPF nástrojov, 
ktoré poskytujú rôzne funkcie pre spravovanie a využívanie eBPF. 
Medzi niektoré z najbežnejších typov eBPF nástrojov patria:
\begin{itemize}
    \item \textbf{Kompilátory eBPF}: Tieto nástroje umožňujú kompilovať eBPF programy zdrojového kódu v jazyku C, aj z jazykov s vyššou úrovňou, 
    ako napríklad Python.
    \item \textbf{Debuggery eBPF}: umožňujú ladenie eBPF programov. Umožňujú napríklad sledovať premenné, 
    prezerať hodnoty máp a tracepointov, zastavovať a spúšťať program na konkrétnych bodoch (angl. \emph{breakpoints}).
    \item \textbf{eBPF nástroje pre sieť}: Existujú nástroje ktoré sú špecializované na sieťovú diagnostiku, monitoring a filtráciu pomocou eBPF. 
    Patria sem napríklad nástroje ako tc, iproute2, bpftrace, bcc a iné. Tieto nástroje umožňujú napríklad zistiť \emph{bottleneck} v sieťovej prevádzke, 
    zaznamenávať zdroj a cieľ sieťovej prevádzky, a umožňujú vykonávať sieťovú filtráciu na úrovni jadra systému.
\end{itemize}

\subsubsection{Kompilátory}
\begin{itemize}
    \item \textbf{LLVM/Clang}: je kompilátor, ktorý využíva technológiu LLVM (\emph{Low Level Virtual Machine}) pre kompiláciu eBPF programov. 
    Clang je C/C++ kompilátor, ktorý je súčasťou LLVM projektu a umožňuje kompilovať eBPF programy napísané v jazyku C/C++.
    \item \textbf{BCC (BPF Compiler Collection)}: Je súbor nástrojov pre kompiláciu eBPF programov, ktorý poskytuje rôzne možnosti pre kompiláciu eBPF 
    programov z jazykov s vyššou úrovňou, ako Python a Lua.~\cite{bcc}
    \item \textbf{Bpftool}: Tento nástroj je súčasťou jadra Linuxu a umožňuje kompilovať eBPF programy, spravovať mapy eBPF, 
    inštalovať a odstraňovať eBPF programy z jadra.
\end{itemize}
Každý eBPF program musí prejsť overovačom  nato aby mohol byť spustený. O to sa stará \emph{eBPF verifier} 
ktorý je taktiež súčasťou jadra Linuxu a umožňuje overiť bezpečnosť eBPF programov skôr než sa spustia v jadre. 
Toto zabezpečuje, že eBPF program neprinesie nežiaduce zmeny v jadre a nebude sa spúšťať nekontrolovateľne.

\subsubsection{Debuggery}
\begin{itemize}
    \item \textbf{bpftrace}: je nástroj pre dynamické sledovanie eBPF programov, ktorý umožňuje okrem sledovania aj ladenie 
    eBPF programov v reálnom čase. Bpftrace umožňuje napríklad sledovať systémové volanie (\emph{syscall}), kontext vstupu/výstupu, 
    prístupy k eBPF mapám a podobne.~\cite{bpftrace}
    \item \textbf{bpf\_dbg}: je otvorený-zdrojový eBPF debugger, ktorý umožňuje prezeranie a ladenie eBPF programov v reálnom čase.
    \item \textbf{BCC}: obsahuje nástroj pre ladenie eBPF kódu a umožňuje sledovať a analyzovať eBPF program v reálnom čase.
\end{itemize}
Programovanie eBPF je výkonné, ale aj zložité. Z toho dôvodu vzniklo niekoľko projektov a dodávateľov, ktorí stavajú na platforme eBPF 
s cieľom vytvoriť novú generáciu nástrojov, ktoré budú pokrývať monitorovanie, bezpečnosť, sieťovanie a ďalej.

\subsection{Knižnice eBPF} 
eBPF knižnice predstavujú sadu funkcií a rozhraní, ktoré uľahčujú prácu s eBPF, poskytujúc rôzne funkcie pre vývoj a správu eBPF programov.~\cite{eBPF} 
Tieto knižnice umožňujú vývojárom a administrátorom systémov pracovať s eBPF bez potreby hlbokých znalostí jadra Linuxu, pričom poskytujú 
silné a flexibilné možnosti pre prácu s eBPF programami.

\subsubsection{Knižnica libbpf}
je generická knižnica eBPF založená na jazyku C/C++, ktorá pomáha oddeliť načítavanie objektových súborov eBPF generovaných kompilátorom clang/LLVM do jadra a vo 
všeobecnosti abstrahuje interakciu so systémovým volaním BPF poskytovaním ľahko použiteľných API knižníc pre aplikácie.~\cite{eBPF} Dôležitá poznámka k tejto knižnici. 
Keďže obsahuje informácie o formáte typu BPF (\emph{BTF}). Je potrebné použiť konfiguráciu jadra, ktoré má pri kompilácii nastavené \texttt{CONFIG\_DEBUG\_INFO\_BTF=y}. 
V prípade, že vaša distribúcia Linuxu nemá toto nastavenie povolené je potrebná manuálna konfigurácia.

\subsubsection{Knižnica BCC}
BCC je sada nástrojov na vytváranie efektívnych programov na sledovanie a manipuláciu s jadrom.~\cite{bcc}  Väčšina funkcií ktorú BCC používa, vyžaduje Linux 4.1 a vyšší. 
BCC uľahčuje písanie programov BPF vďaka inštrumentácii jadra v jazyku C a front-endom v jazykoch Python a lua. Je vhodná na mnohé úlohy vrátane analýzy výkonu a 
riadenia sieťovej prevádzky.

\subsubsection{Knižnica eBPF Go}
eBPF Go je knižnica jazyka Go, ktorá poskytuje nástroje na načítanie, kompiláciu a ladenie programov eBPF. Má minimálne externé závislosti 
a je určená na používanie v dlhodobo bežiacich procesoch. 

\subsection{Použitie eBPF}
eBPF programy môžu byť použité pre rôzne účely, ako napríklad pre sieťovú diagnostiku, sledovanie výkonu, filtrovanie sieťovej aktivity, 
a implementáciu rôznych sieťových a bezpečnostných politík. Táto časť je zameraná na to kde sa dá eBPF použiť a aké bezpečnostné riziká, 
toto použitie prináša.
\begin{itemize}
\item \textbf{Monitoring systému}: eBPF sa často používa na sledovanie systémových metrík ako je využitie CPU, pamäte a sieťovej komunikácie.
\item \textbf{Sledovanie kódu}: eBPF sa môže použiť na sledovanie a profilovanie kódu v reálnom čase.
\item \textbf{Ochrana pred útokmi}: eBPF sa môže použiť na detekciu a blokovanie škodlivého kódu v reálnom čase.
\item \textbf{Firewall}: eBPF sa môže použiť ako firewall, ktorý je schopný vykonávať pokročilé filtrovanie sieťového provozu.
\end{itemize}
Pre zabezpečenie operačného systému Linux sú to napríklad:
\begin{itemize}
    \item \textbf{Kontrola sieťovej komunikácie}: eBPF môže byť použitý na filtrovanie sieťovej aktivity, aby sa zabránilo nežiadúcemu prístupu alebo útoku. 
    Môže tiež byť použitý na vyhľadávanie nebezpečných alebo podozrivých sieťových aktivít a upozornenie na ne.
    \item \textbf{Kontrola súborov}: eBPF môže byť použitý na monitorovanie a kontrolu súborov v operačnom systéme, aby sa zabránilo neoprávnenému prístupu alebo útoku.
    \item \textbf{Kontrola procesov} eBPF môže byť použitý na monitorovanie a kontrolu procesov v operačnom systéme, aby sa zabránilo neoprávnenému použitiu 
    alebo útoku na systém. Môže tiež byť použitý na detekciu a blokovanie nebezpečných procesov alebo aktivít.
    \item \textbf{Kontrola pamäte}: eBPF môže byť použitý na monitorovanie a kontrolu používania pamäte v operačnom systéme, aby sa zabránilo zneužitiu 
    alebo útoku na pamäť.
    \item \textbf{Riadenie systémových volaní}: eBPF môže byť použitý na monitorovanie a riadenie systémových volaní v operačnom systéme, 
    aby sa zabránilo neoprávnenému použitiu alebo útoku.
\end{itemize}

\subsubsection{Bezpečnosť a eBPF}
Sekcia je vypracovaná podľa~\cite{eBPF}

Počas vývoja eBPF bola bezpečnosť najdôležitejším aspektom pri zvažovaní začlenenia eBPF do jadra Linuxu. 
Bezpečnostné riziká eBPF súvisia s tým, že eBPF programy môžu spúšťať arbitrárny kód v jadre operačného systému. 
To znamená, že eBPF programy musia byť správne navrhnuté, implementované a testované, aby sa predišlo bezpečnostným problémom. 
Kritickou časťou je validácia eBPF pred spustením, aby sa zabezpečilo, že eBPF program neobsahuje nebezpečný kód. 
Zraniteľnosti v eBPF programoch sú tiež jedným z potenciálnych bezpečnostných rizík. Tieto zraniteľnosti môžu byť využité pre preniknutie do systému, 
napríklad pre získanie prístupu k root právam alebo pre spustenie vlastného škodlivého kódu. Je nevyhnutné preto, 
aby boli eBPF programy pravidelne skúmané na prítomnosť zraniteľností a aby boli rýchlo aktualizované pri jej objavení. 
Ďalším dôležitým aspektom bezpečnosti eBPF je kontrola prístupu k eBPF programom. Je potrebné zabezpečiť, aby iba oprávnené osoby 
mali prístup k eBPF programom a ich konfigurácii. Je tiež dôležité zaznamenávať a sledovať akúkoľvek neoprávnenú aktivitu s eBPF programami. 
V konečnom dôsledku, bezpečnosť eBPF je kritickou témou pre úspešné používanie eBPF. Je nevyhnutné, aby boli eBPF programy 
správne navrhnuté, implementované a testované, a aby bolo zabezpečené ich chránenie pred potenciálnymi bezpečnostnými rizikami.
eBPF bezpečnosť je zabezpečená prostredníctvom niekoľkých vrstiev:
Začneme s požadovanými oprávneniami. Pokiaľ nie je povolený neprivilegovaný eBPF, všetky procesy, ktoré majú v úmysle načítať programy 
eBPF do jadra Linuxu, musia byť spustené v privilegovanom režime (root) alebo musia vyžadovať schopnosť \texttt{CAP\_BPF}. 
To znamená, že nedôveryhodné programy nemôžu načítať programy eBPF. V prípade, že je povolený neprivilegovaný režim eBPF, 
neprivilegované procesy môžu načítať určité programy eBPF s výhradou obmedzenej sady funkcií a s obmedzeným prístupom k jadru. 
Skôr ako bude procesu povolené načítať program eBPF musí tento program prejsť procesom overenia cez takzvaný overovač (angl. \emph{Verifier}). 
Overovač eBPF zabezpečuje bezpečnosť samotného programu. Pri overení sa kontroluje či program neobsahuje funkcie ktoré by mohli spôsobiť prerušenie systému. 
Kontroluje sa: 
\begin{itemize}
    \item Či programy eBPF neobsahujú neinicializované premenné, a taktiež či nenastáva prístup do pamäte mimo hraníc.
    \item Či programy eBPF môžu obsahovať tzv. ohraničené cykly. Program je prijatý len vtedy, ak overovač môže zabezpečiť, 
    že cyklus obsahuje výstupnú podmienku, ktorá sa zaručene stane pravdivou.
    \item Veľkosť programov eBPF, nakoľko je potrebné aby sa programy zmestili do požiadaviek na veľkosť operačného systému. 
    Nie je možné načítať ľubovoľne veľké programy eBPF.
    \item Každý program musí mať konečnú zložitosť. Overovač vyhodnotí všetky možné cesty vykonávania a musí byť schopný dokončiť analýzu 
    v medziach nakonfigurovanej hornej hranice zložitosti.
\end{itemize}

Po úspešnom dokončení overovania program eBPF prejde procesom tvrdenia (angl. \emph{Hardening}) podľa toho, či je program načítaný z 
privilegovaného alebo neprivilegovaného procesu. 
Tento krok zahŕňa:
\begin{itemize}
    \item \textbf{Ochranu vykonávania programu}: 
        \begin{itemize}
            \item Pamäť jadra, v ktorej sa nachádza program eBPF, je chránená a je určená len na čítanie. 
            Pokiaľ sa program pokúsi niečo modifikovať, jadro sa zrúti aby neumožnilo pokračovať vo vykonávaní poškodeného/manipulovaného programu.
        \end{itemize}
    \item \textbf{Zmiernenie proti Spectre}: 
        \begin{itemize}
            \item Pri špekulácii môžu procesory nesprávne predpovedať vetvy a zanechať pozorovateľné vedľajšie efekty, 
            ktoré by sa mohli extrahovať prostredníctvom bočného kanála.
        \end{itemize}
    \item \textbf{Konštantné zaslepenie}: 
        \begin{itemize}
            \item Všetky konštanty v kóde sú zaslepené, aby sa zabránilo útokom JIT spraying.
        \end{itemize}
\end{itemize}

\subsubsection{Seccomp}
Následujúca sekcia je vypracovaná podľa ~\cite{seccomp}

Jedným z prvých použití virtuálneho stroja BPF mimo siete bola implementácia politík kontroly prístupu pre systémové volanie seccomp(). 
\emph{Secure computing mode} alebo seccomp je bezpečnostná funkcia jadra Linux, ktorá umožňuje obmedzenie systémových volaní (\emph{syscalls}), 
ktoré môžu byť vykonávané procesom. Jedná sa o jednoduchý, ale flexibilný mechanizmus sandboxingu, ktorý sa široko používa. 
Účelom programu BPF pod funkciou seccomp() je rozhodovať o tom, či má byť dané systémové volanie povolené. Filtračné programy 
seccomp bežia na "klasickom" virtuálnom stroji BPF, a nie na eBPF. To ale neznamená, že nie je možné skombinovať seccomp s eBPF, 
aj keď v čase písania tejto práce nie je možné úplne integrovať eBPF do systému seccomp kvôli bezpečnostným problémom. 
Prechodom na eBPF by sa seccomp() programom sprístupnilo množstvo nových funkcií vrátane máp, pomocných funkcií, ukladania na jednotlivé úlohy, 
expresívnejšej inštrukčnej sady a ďalších. Kombinácia eBPF s filtrom seccomp je možná, vyžaduje však úpravu predvolených 
nastavení a konfiguráciu, ktorá by sa mala testovať a zohľadniť konkrétne požiadavky a potreby aplikácie predtým, než sa nasadí do produkčného 
prostredia. eBPF môže byť použitý na implementáciu filtrov, ktoré kontrolujú a ovplyvňujú systémové volania, ktoré sú vykonávané procesom. 
Ďalej poskytuje viacúrovňovú ochranu pre systém, pretože eBPF filtre môžu byť použité na hlbšiu kontrolu systémových volaní, 
pokiaľ seccomp filtre poskytnú základnú ochranu pred ostatými nebezpečnými systémovými volaniami. Ochrana systému musí byť komplexná a zahŕňať viacero prvkov 
a implementáciu dobrej bezpečnostnej politiky.

\subsubsection{LSM}
Následujúca sekcia je vypracovaná podľa internetovej stránky~\cite{lsm,lsm-usage}

\emph{Linux Security Modules} v skratke LSM je rozšírenie operačného systému Linux, ktoré umožňuje pridávať rôzne moduly pre zlepšenie zabezpečenia. 
BPF je jedným z týchto modulov, ktorý môže byť použitý na rôzne účely, vrátane zlepšenia bezpečnosti operačného systému. BPF môže byť nakonfigurovaný tak, 
aby filtroval sieťový prenos na základe rôznych kritérií, ako napríklad IP adresy, porty alebo protokoly. Týmto spôsobom môžete zabrániť 
nežiaducim sieťovým prenosom, ako sú napríklad útoky DDoS, a zlepšiť zabezpečenie siete. Ďalším spôsobom, ako je možné použiť LSM BPF 
na zlepšenie bezpečnosti operačného systému, je kontrolou prístupu k systémovým objektom, ako sú súbory alebo procesy. Nastavením eBPF tak, 
aby kontroloval prístup napríklad na základe užívateľského mena alebo skupiny. Týmto spôsobom je možné 
zlepšiť kontrolu prístupu k dôležitým systémovým objektom a zvýšiť bezpečnosť operačného systému. Framework bezpečnostného modulu Linuxu (LSM) 
poskytuje mechanizmus na pripojenie rôznych bezpečnostných kontrol pomocou rozšírení jadra. Primárnymi používateľmi rozhrania LSM 
sú rozšírenia \textbf{MAC} (\emph{Mandatory Access Control}), ktoré poskytujú komplexnú bezpečnostnú politiku. Okrem väčších rozšírení MAC možno pomocou 
rozhrania LSM vytvárať aj ďalšie rozšírenia, ktoré poskytujú špecifické zmeny fungovania systému, ak tieto úpravy nie sú k dispozícii 
v základnej funkcii samotného systému Linux. Z pohľadu bezpečnostného správania sa lepšie mapuje na LSM ako na filtre seccomp, 
ktoré sú založené na zachytávaní systémových volaní. Pri filtri seccomp sa bezpečnostné správanie môže realizovať prostredníctvom viacerých systémových volaní, 
pričom je ľahké jedno alebo viacero z nich prehliadnuť, zatiaľ čo LSM hooks zachytávajú správanie, ktoré je predmetom záujmu. 
Zámerom je, aby pomocné eBPF funkcie boli "presné a granulárne".

\subsubsection{KRSI}
Následujúca sekcia je vypracovaná podľa internetovej stránky~\cite{krsi,krsi-sm}

\emph{Kernel Runtime Security Instrumentation}, alebo v skratke KRSI je koncepcia, ktorá umožňuje vykonávať vlastný kód v jadre operačného systému v reálnom čase, 
čo umožňuje vykonávať rôzne bezpečnostné úlohy. Hlavným cieľom je sledovať celkové správanie systému za účelom odhalenia útokov.  
eBPF je jazyk a platforma pre KRSI, ktorá umožňuje vývojárom naprogramovať vlastné eBPF programy, ktoré môžu byť spustené v jadre operačného systému. 
Tieto programy môžu byť využité na rôzne úlohy, ako napríklad monitorovanie siete, detekciu útokov, bezpečnostnú kontrolu prístupu k súborom, a podobne. 
\subsection*{Ako používať KRSI}
Prototyp KRSI je implementovaný ako bezpečnostný modul Linuxu (LSM), ktorý umožňuje pripojenie programov eBPF k bezpečnostným hookom jadra. 
Pre používanie KRSI, je potrebné mať verziu jadra, ktorá podporuje eBPF. Po načítaní eBPF programov do jadra ich možno pripojiť k rôznym 
funkciám jadra pomocou mechanizmu KRSI. To umožňuje, aby sa programy eBPF spustili vždy, keď sa tieto funkcie zavolajú, čo umožňuje dynamické 
presadzovanie bezpečnostných politík. V prípade zabezpečenia operačného systému sa KRSI dá použiť na monitorovanie systému a zhromažďovanie údajov 
o akomkoľvek podozrivom správaní, ako je napríklad pokus o neoprávnený prístup. KRSI exportuje novú hierarchiu súborového systému 
pod \texttt{/sys/kernel/security/bpf} s jedným súborom pre každý hook. Na daný hook môže byť pripojených viac ako jeden program a pri 
každom volaní sa postupne zavolajú všetky pripojené programy BPF. Ak niektorý program BPF vráti chybový stav, požadovaná akcia sa zamietne.

\section{Aplikácie eBPF}
V sieťovom priemysle sa eBPF používa na monitorovanie a filtrovanie sieťovej premávky, čo umožňuje analýzu a optimalizáciu sieťovej infraštruktúry. 
Tiež sa využíva na implementáciu bezpečnostných opatrení, ako napr. kontrola prístupu k sieťovej premávke, detekcia útokov a ochrana pred útokmi. 
V oblasti kontroly procesov v priemysle, eBPF umožňuje kontrolu a riadenie procesov na úrovni jadra operačného systému, čo poskytuje presné a rýchle sledovanie 
využívania zdrojov a umožňuje lepšiu analýzu výkonu procesov. V oblasti telekomunikácií eBPF sa využíva na monitorovanie a riadenie sieťovej aktivity, 
detekciu a odstránenie \emph{bottleneckov}, ako aj na implementáciu bezpečnostných opatrení pre telekomunikačné siete. Všeobecne je eBPF využívaný v rôznych odvetviach, 
kde potrebujeme kontrolovať a riadiť procesy na úrovni jadra operačného systému alebo poskytovať presné a rýchle sledovanie využívania zdrojov. Taktiež umožňuje lepšiu analýzu výkonu 
a optimalizáciu procesov.

\subsection{Detekcia útokov}
Existujú rôzne spôsoby, ako môže byť eBPF použitý na detekciu útokov, sú to napríklad:
\begin{itemize}
    \item \textbf{Monitorovanie siete}: eBPF môže byť použitý na monitorovanie sieťovej premávky v reálnom čase.
    \item \textbf{Riadenie systémových volaní}: eBPF môže byť použitý na riadenie systémových volaní, ktoré sú vykonávané procesom, a na detekciu neobvyklého využívania 
    systémového volania, ktoré by mohlo byť spôsobené útokom.
    \item \textbf{Analýza údajov}: eBPF môže byť použitý na analyzovanie údajov získaných z monitorovania siete a systémových volaní, a na detekciu 
    neobvyklého správania, ktoré by mohlo byť spôsobené útokom.    
\end{itemize}
Použitie eBPF ako jedného zo zabezpečovacích nástrojov môže byť veľmi efektívne pri detekcii útokov, ale musí 
sa kombinovať s ďalšími opatreniami pre zabezpečenie systému.

\section{Porovnanie eBPF s inými nástrojmi} 
V dnešnej dobe sa môžeme stretnúť s veľkou ponukou nástrojov slúžiacich na zabezpečenie operačného systému. 
Pár najznámejších si uvedieme v zozname nižšie.  
\begin{itemize}
    \item \textbf{SELinux (Security Enhanced Linux)}: je klasický nástroj na zabezpečenie operačného systému Linux, ktorý poskytuje mechanizmy 
    izolácie a ochrany prostredníctvom rolí a politík.
    \item \textbf{AppArmor}: je nástroj pre dynamickú kontrolu prístupu, ktorý umožňuje nastavenie profilov pre jednotlivé aplikácie, 
    aby sa zabránilo nežiadúcemu prístupu k systémovým zdrojom.
    \item \textbf{FirewallD}: je firewall pre Linux, ktorý umožňuje správu firewallu prostredníctvom rozhrania D-Bus.
    \item \textbf{Iptables}: je klasický firewall nástroj pre Linux, ktorý poskytuje základnú kontrolu prístupu k sieťovej premávke.
    \item \textbf{OpenSSH}: je nástroj pre zabezpečenie sieťového pripojenia, ktorý umožňuje bezpečné prihlásenie a komunikáciu 
    prostredníctvom šifrovaných protokolov.
    \item \textbf{Tripwire}: je nástroj na detekciu zmien v súborovom systéme, ktorý umožňuje detegovať nežiaduce zmeny v 
    systéme a prijímať opatrenia na ich odstránenie.
\end{itemize}

\section{Implementácia pomocou eBPF}
Predbežným návrhom na implemetáciu sú malé programy zamerané na:
\begin{enumerate}
    \item \textbf{Kontrolu súborov}: eBPF môže byť použitý na monitorovanie a kontrolu súborov v operačnom systéme, aby sa zabránilo 
    neoprávnenému prístupu alebo útoku.
    \item \textbf{Kontrolu procesov}: eBPF môže byť použitý na monitorovanie a kontrolu procesov v operačnom systéme, aby sa zabránilo 
    neoprávnenému použitiu alebo útoku. Môže tiež byť použitý na detekciu a blokovanie nebezpečných procesov alebo aktivít.
    \item \textbf{Kontrolu pamäte}: eBPF môže byť použitý na monitorovanie a kontrolu používania pamäte v operačnom systéme, aby sa zabránilo zneužitiu 
    alebo útoku na pamäť.
    \item \textbf{Kontrolu systémových volaní}: eBPF môže byť použitý na monitorovanie a kontrolu systémových volaní v operačnom systéme, 
    aby sa zabránilo neoprávnenému použitiu alebo útoku.
    \item \textbf{Ochranu pred fileless malware}: eBPF môže byť použité pre demonštratívnu implementáciu ochrany pred fileless malwarom.
\end{enumerate}

\section{Výsledky implemetácie}











